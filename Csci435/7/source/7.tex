\documentclass[10pt] {article}
\usepackage{fullpage}
\usepackage{amssymb}
\usepackage{graphicx}
\usepackage{float}
\usepackage{amsthm}
\usepackage{hyperref}
\usepackage{amsmath}
\renewcommand\qedsymbol{$\blacksquare$}

\title{Homework 7 }
\author{Ricky Hempel}
\begin{document}
\maketitle
\begin{center}
Chapter 4: Exercises: 4.1-6, 4.9
\end{center}
\begin{enumerate}
\item[4.1]
(a.)Yes the DFA M accepts 0100.\\
(b.)No M does not accept 011.\\
(c.)No this input has only a single component and therefore is not in the correct form.\\
(d.)No the first component is not a regular expression and so the input is not of the correct form.\\
(e.)No M's language is not empty\\
(f.)Yes M accepts the same language as itself.
\item[4.2]\begin{proof}
The language can be express as the set L=$\{ \langle R,S \rangle\ \mid$ R is a DFA and is S is a regex where $L(R)=L(S)\}$.Let the Turning Machine T be the TM that decides L. \\
T="On input $L=\langle R,S \rangle$:\\
1.Convert S to a NFA N.\\
2.Convert N to a DFA D.\\
3. By using Theorem 4.5. Feed R and D into a TM M that decides whether or they are equal or not.\\
(a.) If M says R and D are equal by accepting then accept.\\
(b.) If M says R and D are not equal by rejecting then reject.  "
\end{proof}
\item[4.3]\begin{proof}
Let the Turning Machine T be the TM that decides $ALL_{DFA}$.\\
T="On input $\langle A \rangle$, where A is a DFA.\\
1.Convert A to $A^c$, the DFA that accepts the complement of the language being accepted by A.\\
2.Using Theorem 4.4 as a subroutine, check and see if $L(A^c)= \emptyset$ or not.\\
3.If $L(A^c)=\emptyset$ accept otherwise reject."
\end{proof}
\item[4.4]\begin{proof}
Let the Turning Machine T be the TM that decides $A_{\varepsilon CFG}$.\\
T="On input $\langle G \rangle$ where G is a context free grammar.\\
1.Convert the grammar G to a CFG G' in CNF. \\
2.If $G'$ contains the production $S \rightarrow \varepsilon$ accept, otherwise reject.
"
\end{proof}
\item[4.5]
\begin{proof}
Let $s_1,s_2,...$ be a list of all strings in $\Sigma^*$. The following TM T recognizes $\overline{E_{TM}}$.\\
T="On input $\langle M \rangle$, where M is a TM:\\
1.Repeat the following for i=1,2,3...\\
2.Run M for i steps on each input,$s_1,s_2,...,s_i$.\\
3.If M has accepted any of these, accept.Otherwise reject."
\end{proof}
\item[4.6](a.)No f is not one-to-one because of f(1)=f(3). \\
(b.)No f is not onto because their is not a x$\in$X such that f(x)=10.\\
(c.)No f is not a correspondence because f is not one-to-one and onto.\\
(d.)Yes g is one-to-one.\\
(e.)Yes g is onto.\\
(f.)Yes g is a correspondence because g is one-to-one and onto.
\item[4.9]\begin{proof}
A relation is known as a equivalence relation if it is reflexive, transitive, and symmetric. A same size relation is a equivalence relation if and only if it is reflexive, transitive, and symmetric.First, it is reflexive because the identity function $f(x)=x,\forall x \in A$ is a correspondence $f: A \rightarrow A$. Second it is symmetric because any correspondence has an inverse, which itself is a correspondence. Lastly, it is transitive because the function $f:A \rightarrow B$ is a bijective function from A to B and the function $g: B \rightarrow C$ is a bijective function from B to C. Therefore the composition of two bijective functions f and g is also a bijective function from A to C.  Hence the same size relation is reflexive,symmetric, and transitive so the same size relation is a equivalence relation.
\end{proof}
\end{enumerate} 
\end{document}